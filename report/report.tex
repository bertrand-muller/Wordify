% !TeX spellcheck = en_GB 

\documentclass{tnreport}
%\documentclass[stage2a]{tnreport1} % If you are in 2nd year
%\documentclass[confidential]{tnreport1} % If you are writing confidential report

\def\reportTitle{Wordify} % Titre du mémoire
\def\reportAuthor{Ophélien Amsler \& Bertrand Müller}

\usepackage{lipsum}
\usepackage{subcaption}
\usepackage{adjustbox}
\usepackage{dirtree}
\usepackage{tikz}
\usepackage{multirow}
\usetikzlibrary{trees}
\usepackage{tikz-qtree}
\usepackage{makecell}
\usepackage{hhline}
\usepackage{colortbl}
\usepackage[most]{tcolorbox}
\usepackage{pdfpages}
\usepackage{draftwatermark}
\usepackage[stable]{footmisc}
\usepackage{hhline}
\usepackage{float}
\usepackage{graphicx}
\usepackage{enumitem}

\usetikzlibrary{calc}
\usetikzlibrary{positioning}

\begin{document}

\maketitle

\clearpage

\renewcommand{\baselinestretch}{0.5}\normalsize
\tableofcontents
\renewcommand{\baselinestretch}{1.0}\normalsize

\clearpage

\chapter{Introduction}

In this report, we will introduce a new tool to improve our English skills while playing a game. The purpose of the project is to promote a new digital tool in order to learn English. Thus, our team worked on an innovative tool to enrich our vocabulary. For this project, the idea is to focus on the interactive side of the solution and to provide a non-existent tool. 

Why is this project relevant at the present time ? At present, learning a language is a major challenge in an globalized society. One of the most commonly used language is English. It is an important mean for dialoguing and to be informed. However, the average English language level is uneven across different regions of the world and Europe. Our project is part of the desire to fill these gaps and to allow people to easily talk or to develop professionally. In addition, learning a new language can be tedious and time consuming. Our wish is to reduce this learning time for people having some knowledge in English. 

In order to present our game, we decided to structure this report according to three different parts. The first one presents the game itself and its rules. Through this first part, we want to provide a user documentation to know how to play and how the interface has to be used. The second is centered on a technical description of the game by presenting its architecture, the tools used, the difficulties encountered and the solutions found. The last part of this report describes a typical scenario as to the proposed game. After these three parts, we conclude on the result obtained and on the possible game evolutions. 

This report aims to present our new way of learning English and to highlight the needs in the current education system. 

\begin{center}
	\includegraphics{figures/lets_play}
\end{center} 

\chapter{Game}

\section{Principle}

\section{Rules}

\section{Educational aspect}

\chapter{User documentation}

\chapter{Documentation technique}

\appendix
\part*{Annexes}
\addcontentsline{toc}{part}{Annexes}

\end{document}
